\chapter{Inleiding}

\engels{Collectible Card Games} (CCGs) of \engels{Trading Card Games} (TCGs) zoals \engels{Magic: The Gathering} en \engels{Shadow Era} zijn snel veranderende spellen waaraan meerdere keren per jaar nieuwe kaarten en spelregels toegevoegd worden. Sinds het einde van de jaren '90 bestaan er ook computerspellen gebaseerd op CCGs. Nieuwe versies van die computerspellen worden echter minder snel gereleased dan hun fysieke tegenhangers. E�n van de redenen hiervoor is dat voor elke nieuwe kaart of spelregel nieuwe broncode geschreven moet worden.
\npar
In een ideale wereld waar computers natuurlijke taal volledig begrijpen zouden nieuwe kaarten door de auteurs zelf kunnen toegevoegd worden, enkel gebruik makend van de instructies die al op de kaarten geprint staat. Een oplossing die het midden houdt tussen het schrijven van broncode en het interpreteren van instructies op de kaarten zelf is een domein-specifieke taal (DST). Een domein-specifieke taal is een programmeer taal die speciaal ontwikkeld werd op problemen binnen een goed gedefinieerd probleemdomein op te lossen. In deze masterproef wordt uitgelegd hoe een DST kan helpen om de snelle veranderingen in een collectible card game te kunnen volgen in de digitale versie ervan.
\npar
De inhoud van deze masterproef ziet er als volgt uit: TODO