
\documentclass[11pt,a4paper,oneside]{book}
\usepackage{a4wide}                     % Iets meer tekst op een bladzijde
\usepackage[dutch]{babel}               % Voor nederlandstalige hyphenatie (woordsplitsing)
\usepackage{amsmath}                    % Uitgebreide wiskundige mogelijkheden
\usepackage{amssymb}                    % Voor speciale symbolen zoals de verzameling Z, R...
\usepackage{makeidx}                    % Om een index te maken
\usepackage{url}                        % Om url's te verwerken
\usepackage{graphicx}                   % Om figuren te kunnen verwerken
\usepackage[small,bf,hang]{caption}     % Om de captions wat te verbeteren
\usepackage{xspace}                     % Magische spaties na een commando
\usepackage[latin1]{inputenc}           % Om niet ascii karakters rechtstreeks te kunnen typen
\usepackage{float}                      % Om nieuwe float environments aan te maken. Ook optie H!
\usepackage{flafter}                    % Opdat floats niet zouden voorsteken
\usepackage{listings}                   % Voor het weergeven van letterlijke text en codelistings
\usepackage[round]{natbib}              % Voor auteur-jaar citaties.
\usepackage[nottoc]{tocbibind}					% Bibliografie en inhoudsopgave in ToC; zie tocbibind.dvi
\usepackage{eurosym}                    % om het euro symbool te krijgen
\usepackage{textcomp}                   % Voor onder andere graden celsius
\usepackage{fancyhdr}                   % Voor fancy headers en footers
\usepackage[Gray,squaren,thinqspace,thinspace]{SIunits} % Om elegant eenheden te zetten
\usepackage[version=3]{mhchem}          % Voor elegante scheikundige formules

% Volgend package is niet echt nodig. Het laat echter toe om gemakkelijk elektronisch
% te navigeren in je pdf-document. Deze package moet altijd als laatste ingeladen worden.
\usepackage[a4paper,plainpages=false]{hyperref}    % Om hyperlinks te hebben in het pdfdocument.


%%%%%%%%%%%%%%%%%%%%%%%%%%%%%%
% Algemene instellingen van het document.
%%%%%%%%%%%%%%%%%%%%%%%%%%%%%%

% De splitsingsuitzonderingen
\hyphenation{back-slash split-sings-uit-zon-de-ring}

%\bibpunct{(}{)}{;}{y}{,}{,}             % Auteur-jaar citaties -- zie natbib.dvi voor meer uitleg; niet echt nodig

% Het bibliografisch opmaak bestand.
% ZORG ERVOOR DAT bibliodutch.bst ZICH IN JE WERKDIRECTORY BEVINDT!!!
\bibliographystyle{bibliodutch}

\setlength{\parindent}{0cm}             % Inspringen van eerste lijn van paragrafen is niet gewenst.

\renewcommand{\baselinestretch}{1.2} 		% De interlinie afstand wat vergroten.

\graphicspath{{figuren/}}               % De plaats waar latex zijn figuren gaat halen.

\makeindex                              % Om een index te genereren.

\setcounter{MaxMatrixCols}{20}          % Max 20 kolommen in een matrix

% De headers die verschijnen bovenaan de bladzijden, herdefinieren:
\pagestyle{fancy}                       % Om aan te geven welke bladzijde stijl we gebruiken.
\fancyhf{}                              % Resetten van al de fancy settings.
\renewcommand{\headrulewidth}{0pt}      % Geen lijn onder de header. Zet dit op 0.4pt voor een mooie lijn.
\fancyhf[HL]{\nouppercase{\textit{\leftmark}}} % Links in de header zetten we de leftmark,
\fancyhead[HR]{\thepage}                % Rechts in de header het paginanummer.
% Activeer de volgende lijn en desactiveer de vorige om paginanummers onderaan gecentreerd te krijgen.
%\fancyhf[FC]{\thepage}                  % Paginanummers onderaan gecentreerd.

% PDF specifieke opties, niet strict noodzakelijk voor een thesis.
% Is hetgeen verschijnt wanneer je in acroread de documentproperties bekijkt.
\hypersetup{
    pdfauthor = {Kenneth Saey},
    pdftitle = {Een domein-specifieke taal in Scala voor collectible card games},
    pdfsubject = {Masterproef ingediend tot het behalen van de academische graad van Master in de ingenieurswetenschappen: computerwetenschappen},
    pdfkeywords = {Masterproef, thesis, Scala, Collectible card games, Domein-specifieke taal}
}


% Het volgende commando zou ervoor moeten zorgen dat er een witte ruimte wordt gelaten tussen
% elke paragraaf. Het zorgt ervoor dat er echter teveel witte ruimte komt boven en onder de
% verschillende titels, gemaakt met \section, subsection...
%%\setlength{\parskip}{0ex plus 0.3ex minus 0.3ex}

% Vandaar dat we expliciet aangeven wanneer we wensen dat een nieuwe paragraaf begint:
% \par zorgt ervoor dat er een nieuwe paragraaf begint en
% \vspace zorgt voor verticale ruimte.
\newcommand{\npar}{\par \vspace{2.3ex plus 0.3ex minus 0.3ex}}

%%%%%%%%%%%%%%%%%%%%%%%%%%%%%%
% Nieuwe commando's
%%%%%%%%%%%%%%%%%%%%%%%%%%%%%%

% De differentiaal operator
\newcommand{\diff}{\ensuremath{\mathrm{d}}} 

% Super en subscript
\newcommand{\supsc}[1]{\ensuremath{^{\text{#1}}}}   % Superscript in tekst
\newcommand{\subsc}[1]{\ensuremath{_{\text{#1}}}}   % Subscript in tekst

% Chemische formule font:
\newcommand{\ch}[1]{\ensuremath{\mathrm{#1}}\xspace}	 
% Chemische pijl naar rechts:
\newcommand{\chpijlr}{\ensuremath{\hspace{1em}\longrightarrow\hspace{1em}}}
% Chemische pijl naar links:
\newcommand{\chpijll}{\ensuremath{\hspace{1em}\longleftarrow\hspace{1em}}}
% Chemische pijl naar links en rechts:
\newcommand{\chpijllr}{\ensuremath{\hspace{1em}\longleftrightarrow\hspace{1em}}}

\newcommand{\vt}[1]{\ensuremath{\boldsymbol{#1}}} % vector in juiste lettertype
\newcommand{\mx}[1]{\ensuremath{\mathsf{#1}}}	  % matrix in juiste lettertype

% Het latex logo in een eenvoudiger commando steken
\newcommand{\latex}{\LaTeX\xspace}

% Het BibTeX logo
\newcommand{\bibtex}{\textsc{Bib}\TeX\xspace}

% Niew commando om bestandnamen anders weer te geven
\newcommand{\bestand}[1]{\lstinline[basicstyle=\sl]{#1}\xspace}

% Niew commando om commando tekst weer te geven
\newcommand{\command}[1]{\lstinline[basicstyle=\tt]{#1}\xspace}
\newcommand{\commandx}[1]{\index{#1}\lstinline[basicstyle=\tt]{#1}\xspace}

%\lstset{morecomment={\%}}
% Commando om latex commando`s weer te geven (x: voor indexing)
%\newcommand{\lcommand}[1]{\lstinline[basicstyle={\tt},{language=[LaTeX]TeX}]{#1}\xspace}
\newcommand{\lcommand}[1]{\lstinline[basicstyle={\tt}]{#1}\xspace}
\newcommand{\lcommandx}[1]{\index{#1}\lstinline[basicstyle=\tt]{#1}\xspace}


% Niew commando om vreemde taal weer te geven (hint: dit commando kan gebruikt
%   worden om latijnse namen, die ook cursief moeten staan, weer te geven.
\newcommand{\engels}[1]{\textit{#1}\xspace}
\newcommand{\engelsx}[1]{\index{#1}\textit{#1}\xspace}

% Niew commando om iets te benadrukken en tegelijkertijd in de index te steken.
\newcommand{\begrip}[1]{\index{#1}\textbf{#1}\xspace}

% Nieuw commando om figuren in te voegen. Gebruik:
% \mijnfiguur[H]{width=5cm}{bestandsnaam}{Het bijschrift bij deze figuur}
\newcommand{\mijnfiguur}[4][ht]{            % Het eerste argument is standaar `ht'.
    \begin{figure}[#1]                      % Beginnen van de figure omgeving
        \begin{center}                      % Beginnen van de center omgeving
            \includegraphics[#2]{#3}        % Het eigenlijk invoegen van de figuur (2: opties, 3: bestandsnaam)
            \caption{#4\label{#3}}          % Het bijschrift (argument 4) en het label (argument 3)
        \end{center}
    \end{figure}
    }

% Nieuw commando om figuren in te voegen. Gebruik:
% \mijnfiguur[H]{bestand-tabular}{Het bijschrift bij deze tabel}    
\newcommand{\mijntabel}[3][ht]{             % Het eerste argument is standaar `ht'.
    \begin{table}[#1]                       % Beginnen van de table omgeving
        \begin{center}                      % Beginnen van de center omgeving
            \caption{#3\label{#2}}          % Het bijschrift (argument 3) en het label (argument 2)
            \input{#2}                      % Invoer van de tabel
        \end{center}
    \end{table}
    }

%%%%%%%%%%%%%%%%%%%%%%%%%%%%%%
% Nieuwe wiskunde operatoren
%%%%%%%%%%%%%%%%%%%%%%%%%%%%%%

\DeclareMathOperator{\integ}{Integraal}

%%%%%%%%%%%%%%%%%%%%%%%%%%%%%%
% Nieuwe omgevingen
%%%%%%%%%%%%%%%%%%%%%%%%%%%%%%

% Een soort theorem omgeving
\newtheorem{levensles}{Levensles}[chapter]

% Om minder belangrijke delen iets kleiner te zetten.
\newenvironment{MinderBelangrijk}{\small}{}

% Een nieuwe omgeving om letterlijke latex tekst weer te geven.
\lstnewenvironment{llt} 
    {
    \vspace{1.2ex plus 0.5ex minus 0.5ex}   % Beetje ruimte voor de letterlijke tekst
    \lstset{                                % Enkele opties:
        basicstyle={\small\tt},             % Iets kleiner
        %language=[LaTeX]{TeX},              % Syntax highlighting
        stepnumber=0,                       % De lijnen worden niet genummerd
        breaklines=true,                    % Als een lijn te lang is, wordt hij afgebroken
        basewidth={0.5em},                  % Breedte van een letter
        xleftmargin=1em}                    % Inspringing van de linker marge
    }
    {\vspace{0.9ex plus 0.5ex minus 0.5ex}  % Beetje ruimte na de letterlijke tekst
    }

% Een nieuwe omgeving om algemene letterlijke tekst weer te geven.
\lstnewenvironment{lt} 
    {
    \vspace{1.2ex plus 0.5ex minus 0.5ex}   % Beetje ruimte voor de letterlijke tekst
    \lstset{                                % Enkele opties:
        basicstyle={\small\tt},             % Iets kleiner en typmachine lettertype
        stepnumber=0,                       % De lijnen worden niet genummerd
        breaklines=true,                    % Als een lijn te lang is, wordt hij afgebroken
        basewidth={0.5em},                  % Breedte van een letter
        xleftmargin=1em}                    % Inspringing van de linker marge
    }
    {\vspace{0.9ex plus 0.5ex minus 0.5ex}  % Beetje ruimte na de letterlijke tekst
    }



%%%%%%%%%%%%%%%%%%%%%%%%%%%%%%
% Einde van de preamble.
% Begin van de body:
%%%%%%%%%%%%%%%%%%%%%%%%%%%%%%

\begin{document}

\frontmatter

%\input{Masterproef titel}                            % Algemene versie, voorzien door Universiteit Gent


% Typisch copyright voor een thesis.
% Te plaatsen juist na het titelblad.

\rule[-0.4\baselineskip]{0cm}{10\baselineskip}   
\par \vspace{2.3ex plus 0.3ex minus 0.3ex}
De auteur en promotor geven de toelating deze scriptie voor consultatie beschikbaar te stellen en delen ervan te kopi�ren voor persoonlijk gebruik. Elk ander gebruik valt onder de beperkingen van het auteursrecht, in het bijzonder met betrekking tot de verplichting uitdrukkelijk de bron te vermelden bij het aanhalen van resultaten uit deze scriptie.
\par \vspace{2.3ex plus 0.3ex minus 0.3ex}
The author and promoter give the permission to use this thesis for consultation and to copy parts of it for personal use. Every other use is subject to the copyright laws, more specifically the source must be extensively specified when using from this thesis.
\par \vspace{2.3ex plus 0.3ex minus 0.3ex}
Gent, Juni 2013 % Vul de juiste datum in!!!
\par \vspace{2.3ex plus 0.3ex minus 0.3ex}

De promotor \hfill De begeleider \hfill De auteur
\npar
\vspace{2cm}
\npar
% Pas de volgende lijn aan!!!
Prof. dr. ir. T. Schrijvers \hfill Benoit Desouter \hfill Kenneth Saey

\thispagestyle{empty} 

					% Copyright voor thesis

\newpage
%\mbox{}\vspace{-1cm}
%\thispagestyle{plain}

%\textbf{\Huge{Woord vooraf}}

\chapter*{Woord vooraf}

%\vspace{2cm}

\begin{slshape}

%\small
Todo
\npar


\vspace{4ex}

\hfill Kenneth Saey

\hfill Gent 3 Juni 2013

%\normalsize
\end{slshape}

                      % Algemene versie

% Voor een echte thesis, komt hier de samenvatting...
%\input{samenvatting}                       % ...in het Nederlands en...
%\input{summary}                            % ...in het Engels.

% De lijnen van de inhoudsopgave iets dichter op elkaar, niet echt nodig voor de thesis, maar 
% voor dit werk kregen we anders een laatste bladzijde met 3 items op.
%\renewcommand{\baselinestretch}{1.08} 	% De interlinie afstand wat vergroten.
%\small\normalsize                       % Nodig om de baselinestretch goed te krijgen.
\tableofcontents
%\renewcommand{\baselinestretch}{1.2} 	% De interlinie afstand wat vergroten.
%\small\normalsize                       % Nodig om de baselinestretch goed te krijgen.

\mainmatter

% De verschillende hoofdstukken:
\input{Introductie}
\input{Hoofdstuk 1 - Documentstructuur}


% De appendices:
\appendix

%
\chapter{Wiskundige symbolen}\label{wisksymb}

Hieronder geven we tabellen met verschillende wiskundige symbolen. Aangepast overgenomen uit de documentatie van \lcommand{amsmath}.\footnote{Namelijk: \bestand{/usr/share/doc/texmf/latex/general/symbols.dvi.gz}} 

\begin{table}[hbp]\begin{center}
\caption{Kleine Griekse letters}
\vspace{1ex}
\begin{tabular}{ll@{\hspace{1cm}}ll@{\hspace{1cm}}ll}
$\alpha      $& \lcommand{\\alpha}      &$\iota   $& \lcommand{\\iota}  &$\varrho   $& \lcommand{\\varrho}   \\
$\beta       $& \lcommand{\\beta}       &$\kappa  $& \lcommand{\\kappa} &$\sigma    $& \lcommand{\\sigma}    \\
$\gamma      $& \lcommand{\\gamma}      &$\lambda $& \lcommand{\\lambda}&$\varsigma $& \lcommand{\\varsigma} \\
$\delta      $& \lcommand{\\delta}      &$\mu     $& \lcommand{\\mu}    &$\tau      $& \lcommand{\\tau}      \\
$\epsilon    $& \lcommand{\\epsilon}    &$\nu     $& \lcommand{\\nu}    &$\upsilon  $& \lcommand{\\upsilon}  \\
$\varepsilon $& \lcommand{\\varepsilon} &$\xi     $& \lcommand{\\xi}    &$\phi      $& \lcommand{\\phi}      \\
$\zeta       $& \lcommand{\\zeta}       &$o       $& \lcommand{o}       &$\varphi   $& \lcommand{\\varphi}   \\
$\eta        $& \lcommand{\\eta}        &$\pi     $& \lcommand{\\pi}    &$\chi      $& \lcommand{\\chi}      \\
$\theta      $& \lcommand{\\theta}      &$\varpi  $& \lcommand{\\varpi} &$\psi      $& \lcommand{\\psi}      \\
$\vartheta   $& \lcommand{\\vartheta}   &$\rho    $& \lcommand{\\rho}   &$\omega    $& \lcommand{\\omega}    
\end{tabular}
\end{center}\end{table}
 
\begin{table}[hbp]\begin{center}
\caption{Griekse hoofdletters, als ze verschillend zijn van de Latijnse. Om ze cursief te krijgen, moet er een \lcommand{var} voorgeplaatst worden. Bijvoorbeeld `$\backslash$varPi' geeft $\varPi$ in plaats van $\Pi$.}
\vspace{1ex}
\begin{tabular}{ll@{\hspace{1cm}}ll@{\hspace{1cm}}ll}
$\Gamma  $& \lcommand{\\Gamma}  &$\Xi      $& \lcommand{\\Xi}       &$\Phi   $& \lcommand{\\Phi}    \\
$\Delta  $& \lcommand{\\Delta}  &$\Pi      $& \lcommand{\\Pi}       &$\Psi   $& \lcommand{\\Psi}    \\
$\Theta  $& \lcommand{\\Theta}  &$\Sigma   $& \lcommand{\\Sigma}    &$\Omega $& \lcommand{\\Omega}  \\
$\Lambda $& \lcommand{\\Lambda} &$\Upsilon $& \lcommand{\\Upsilon}                       
\end{tabular}
\end{center}\end{table}
 
\begin{table}[hbp]\begin{center}
\caption{Andere symbolen}\label{tab:mathoth}
\vspace{1ex}
\begin{tabular}{ll@{\hspace{1cm}}ll@{\hspace{1cm}}ll}
$\aleph   $& \lcommand{\\aleph}   &$\prime     $& \lcommand{\\prime}     &$\forall      $& \lcommand{\\forall}     \\
$\hbar    $& \lcommand{\\hbar}    &$\emptyset  $& \lcommand{\\emptyset}  &$\exists      $& \lcommand{\\exists}     \\
$\imath   $& \lcommand{\\imath}   &$\nabla     $& \lcommand{\\nabla}     &$\neg         $& \lcommand{\\neg}        \\
$\jmath   $& \lcommand{\\jmath}   &$\surd      $& \lcommand{\\surd}      &$\flat        $& \lcommand{\\flat}       \\
$\ell     $& \lcommand{\\ell}     &$\top       $& \lcommand{\\top}       &$\natural     $& \lcommand{\\natural}    \\
$\wp      $& \lcommand{\\wp}      &$\bot       $& \lcommand{\\bot}       &$\sharp       $& \lcommand{\\sharp}      \\
$\Re      $& \lcommand{\\Re}      &$\Vert      $& \lcommand{\\Vert}      &$\clubsuit    $& \lcommand{\\clubsuit}   \\
$\Im      $& \lcommand{\\Im}      &$\angle     $& \lcommand{\\angle}     &$\diamondsuit $& \lcommand{\\diamondsuit}\\
$\partial $& \lcommand{\\partial} &$\triangle  $& \lcommand{\\triangle}  &$\heartsuit   $& \lcommand{\\heartsuit}  \\
$\infty   $& \lcommand{\\infty}   &$\backslash $& \lcommand{\\backslash} &$\spadesuit   $& \lcommand{\\spadesuit}
\end{tabular}
\end{center}\end{table}

\begin{table}[hbp]\begin{center}
\caption{Verzamelingsoperatoren}
\vspace{1ex}
\begin{tabular}{ll@{\hspace{1cm}}ll@{\hspace{1cm}}ll}
$\sum$ & \lcommand{\\sum} &$\bigcap$ & \lcommand{\\bigcap} &
  $\bigodot$ & \lcommand{\\bigodot} \\
$\prod$ & \lcommand{\\prod} &$\bigcup$ & \lcommand{\\bigcup} &
  $\bigotimes$ & \lcommand{\\bigotimes} \\
$\coprod$ & \lcommand{\\coprod} &$\bigsqcup$ & \lcommand{\\bigsqcup} &
  $\bigoplus$ & \lcommand{\\bigoplus} \\
$\int$ & \lcommand{\\int} &$\bigvee$ & \lcommand{\\bigvee} &
  $\biguplus$ & \lcommand{\\biguplus} \\
$\oint$ & \lcommand{\\oint} &$\bigwedge$ & \lcommand{\\bigwedge}
\end{tabular} 
\end{center}\end{table}

\begin{table}[hbp]\begin{center}
\caption{Binaire operatoren}
\vspace{1ex}
\begin{tabular}{ll@{\hspace{1cm}}ll@{\hspace{1cm}}ll}
$+$   & \verb}+}   &$-$    & \verb}-}    \\
$\pm $& \lcommand{\\pm} &$\cap $& \lcommand{\\cap} &$\vee $& \lcommand{\\vee} \\
$\mp $& \lcommand{\\mp} &$\cup $& \lcommand{\\cup} &$\wedge $& \lcommand{\\wedge} \\
$\setminus $& \lcommand{\\setminus} &$\uplus $& \lcommand{\\uplus} &
$\oplus $& \lcommand{\\oplus} \\
$\cdot $& \lcommand{\\cdot} &$\sqcap $& \lcommand{\\sqcap} &
$\ominus $& \lcommand{\\ominus} \\
$\times $& \lcommand{\\times} &$\sqcup $& \lcommand{\\sqcup} &
$\otimes $& \lcommand{\\otimes} \\
$\ast $& \lcommand{\\ast} &$\triangleleft $& \lcommand{\\triangleleft} &
$\oslash $& \lcommand{\\oslash} \\
$\star $& \lcommand{\\star} &$\triangleright $& \lcommand{\\triangleright} &
$\odot $& \lcommand{\\odot} \\
$\diamond $& \lcommand{\\diamond} &$\wr $& \lcommand{\\wr} &
$\dagger $& \lcommand{\\dagger} \\
$\circ $& \lcommand{\\circ} &$\bigcirc $& \lcommand{\\bigcirc} &
$\ddagger $& \lcommand{\\ddagger} \\
$\bullet $& \lcommand{\\bullet} &$\bigtriangleup $& \lcommand{\\bigtriangleup} &
$\amalg $& \lcommand{\\amalg} \\
$\div $& \lcommand{\\div} &$\bigtriangledown $& \lcommand{\\bigtriangledown} &
\end{tabular}
\end{center}\end{table}

\begin{table}[hbp]\begin{center}
\caption{Relationele operatoren. Merk op dat de negatie van een symbool altijd kan verkregen worden door er $\mathrm{\backslash not}$ voor te zetten: $\mathrm{\backslash not\backslash in}$ geeft $\not\in$.}
\vspace{1ex}
\begin{tabular}{ll@{\hspace{1cm}}ll@{\hspace{1cm}}ll}
$< $& \verb}<} &$>$& \verb}>} &$=$& \verb}=} \\
$\leq $& \lcommand{\\leq} &$\geq $& \lcommand{\\geq} &$\equiv $& \lcommand{\\equiv} \\
$\prec $& \lcommand{\\prec} &$\succ $& \lcommand{\\succ} &$\sim $& \lcommand{\\sim} \\
$\preceq $& \lcommand{\\preceq} &$\succeq $& \lcommand{\\succeq} &
$\simeq $& \lcommand{\\simeq} \\
$\ll $& \lcommand{\\ll} &$\gg $& \lcommand{\\gg} &$\asymp $& \lcommand{\\asymp} \\
$\subset $& \lcommand{\\subset} &$\supset $& \lcommand{\\supset} &
$\approx $& \lcommand{\\approx} \\
$\subseteq $& \lcommand{\\subseteq} &$\supseteq $& \lcommand{\\supseteq} &
$\cong $& \lcommand{\\cong} \\
$\bowtie $& \lcommand{\\bowtie} \\
$\in $& \lcommand{\\in} &$\ni $& \lcommand{\\ni} &$\propto $& \lcommand{\\propto} \\
$\vdash $& \lcommand{\\vdash} &$\dashv $& \lcommand{\\dashv} &
$\models $& \lcommand{\\models} \\
$\smile $& \lcommand{\\smile} &$\mid $& \lcommand{\\mid} &
$\doteq $& \lcommand{\\doteq} \\
$\frown $& \lcommand{\\frown} &$\parallel $& \lcommand{\\parallel} &
$\perp $& \lcommand{\\perp} \\
$\not< $& \lcommand{\\not<} &$\not> $& \lcommand{\\not>} &$\not= $& \lcommand{\\not=} \\
%$\not\leq $& \lcommand{\\not\\leq} &$\not\geq $& \lcommand{\\not\\geq} &$\not\equiv $& \lcommand{\\not\\equiv} \\
$\cdots$ &&$\cdots$ && $\cdots$ 
\end{tabular}
\end{center}\end{table}

%\begin{table}[hbp]\begin{center}
%\caption{Negaties}
%\vspace{1ex}
%\begin{tabular}{ll@{\hspace{1cm}}ll@{\hspace{1cm}}ll}
%$\not< $& \lcommand{\\not<} &$\not> $& \lcommand{\\not>} &$\not= $& \lcommand{\\not=} \\
%$\not\leq $& \lcommand{\\not\\leq} &$\not\geq $& \lcommand{\\not\\geq} &$\not\equiv $& \lcommand{\\not\\equiv} \\
%$\not\prec $& \lcommand{\\not\\prec} &$\not\succ $& \lcommand{\\not\\succ} &$\not\sim $& \lcommand{\\not\\sim} \\
%$\not\preceq $& \lcommand{\\not\\preceq} &$\not\succeq $& \lcommand{\\not\\succeq} &$\not\simeq $& \lcommand{\\not\\simeq} \\
%$\not\subset $& \lcommand{\\not\\subset} &$\not\supset $& \lcommand{\\not\\supset} &$\not\approx $& \lcommand{\\not\\approx} \\
%$\not\subseteq $& \lcommand{\\not\\subseteq} &$\not\supseteq $&\lcommand{\\not\\supseteq} &$\not\cong $& \lcommand{\\not\\cong} \\
%$\not\sqsubseteq $& \lcommand{\\not\\sqsubseteq} &$\not\sqsupseteq $&\lcommand{\\not\\sqsupseteq} &$\not\asymp $& \lcommand{\\not\\asymp} \\
%\end{tabular}
%\end{center}\end{table}

\begin{table}[hbp]\begin{center}
\caption{Pijlen}
\vspace{1ex}
\begin{tabular}{ll@{\hspace{1cm}}ll@{\hspace{1cm}}ll}
$\leftarrow         $& \lcommand{\\leftarrow}         &$\longleftarrow      $& \lcommand{\\longleftarrow}  &$\uparrow     $& \lcommand{\\uparrow}     \\
$\Leftarrow         $& \lcommand{\\Leftarrow}         &$\Longleftarrow      $&\lcommand{\\Longleftarrow}   &$\Uparrow     $& \lcommand{\\Uparrow}     \\
$\rightarrow        $& \lcommand{\\rightarrow}        &$\longrightarrow     $&\lcommand{\\longrightarrow}  &$\downarrow   $& \lcommand{\\downarrow}   \\
$\Rightarrow        $& \lcommand{\\Rightarrow}        &$\Longrightarrow     $&\lcommand{\\Longrightarrow}  &$\Downarrow   $& \lcommand{\\Downarrow}   \\
$\leftrightarrow    $& \lcommand{\\leftrightarrow}    &$\longleftrightarrow $&\lcommand{\\longleftrightarrow}&$\updownarrow$& \lcommand{\\updownarrow}\\
$\Leftrightarrow    $& \lcommand{\\Leftrightarrow}    &$\Longleftrightarrow $&\lcommand{\\Longleftrightarrow}&$\Updownarrow$& \lcommand{\\Updownarrow}\\
$\mapsto            $& \lcommand{\\mapsto}            &$\longmapsto         $& \lcommand{\\longmapsto}     &$\nearrow     $& \lcommand{\\nearrow}     \\
$\hookleftarrow     $& \lcommand{\\hookleftarrow}     &$\hookrightarrow     $&\lcommand{\\hookrightarrow}  &$\searrow     $& \lcommand{\\searrow}     \\
$\leftharpoonup     $& \lcommand{\\leftharpoonup}     &$\rightharpoonup     $&\lcommand{\\rightharpoonup}  &$\swarrow     $& \lcommand{\\swarrow}     \\
$\leftharpoondown   $& \lcommand{\\leftharpoondown}   &$\rightharpoondown   $&\lcommand{\\rightharpoondown}&$\nwarrow     $& \lcommand{\\nwarrow}     \\
$\rightleftharpoons $& \lcommand{\\rightleftharpoons} &
\end{tabular}
\end{center}\end{table}

\begin{table}[hbp]\begin{center}
\caption{Haakjes. Rechterhaakjes kunnen bekomen worden door de eerste \lcommand{l} te vervangen door een \lcommand{r}.\label{haakjestabel}}
\vspace{1ex}
\begin{tabular}{ll@{\hspace{1cm}}ll@{\hspace{1cm}}ll@{\hspace{1cm}}ll@{\hspace{1cm}}ll}
$(       $& \lcommand{(}         &$[          $& \lcommand{\[}          &$\{$           & \lcommand{\\\{}         &$\lceil  $& \lcommand{\\lceil}    &$\lvert  $& \lcommand{\\lvert}\\
$\langle $& \lcommand{\\langle}  &$\lbrack    $& \lcommand{\\lbrack}    &$\lbrace      $& \lcommand{\\lbrace}     &$\lfloor $& \lcommand{\\lfloor}   &$\lVert  $& \lcommand{\\lVert}\\
%$\uparrow$& \lcommand{\\uparrow} &$\downarrow $& \lcommand{\\downarrow} &$\updownarrow $&\lcommand{\\updownarrow} &$\lvert  $& \lcommand{\\lvert}    \\
%$\Uparrow$& \lcommand{\\Uparrow} &$\Downarrow $& \lcommand{\\Downarrow} &$\Updownarrow $&\lcommand{\\Updownarrow} &$\lVert  $& \lcommand{\\lVert} 
\end{tabular} 
\end{center}\end{table}

%\begin{table}[hbp]\begin{center}
%\caption{Bijzondere constructies}
%\vspace{1ex}
%\begin{tabular}{ll@{\hspace{1cm}}ll}
%$\widetilde{abc}     $ & \lcommand{\\widetilde\{abc\}}     & $\widehat{abc}        $ & \lcommand{\\widehat\{abc\}}        \\ 
%$\overleftarrow{abc} $ & \lcommand{\\overleftarrow\{abc\}} & $\overrightarrow{abc} $ & \lcommand{\\overrightarrow\{abc\}} \\ 
%$\overline{abc}      $ & \lcommand{\\overline\{abc\}}      & $\underline{abc}      $ & \lcommand{\\underline\{abc\}}      \\ 
%$\overbrace{abc}     $ & \lcommand{\\overbrace\{abc\}}     & $\underbrace{abc}     $ & \lcommand{\\underbrace\{abc\}}     \\
%$\sqrt{abc}          $ & \lcommand{\\sqrt\{abc\}}          & $\sqrt[n]{abc}        $ & \lcommand{\\sqrt[n]\{abc\}}        \\
%$f'                  $ & \lcommand{f'}                     & $\frac{abc}{xyz}      $ & \lcommand{\\frac\{abc\}\{xyz\}}      
%\end{tabular}
%\end{center}\end{table}



% De bibliografie en de index
\backmatter

\bibliography{Masterproef bibliografie}

%\printindex                             % Om de index af te printen, niet bij een thesis.

\end{document}

