Collectible Card Games (CCG), also known as Trading Card Games (TCG), like \textit{Magic: The Gathering}, \textit{Shadow Era} and pre-teen versions like \textit{Pok\'eMon} are fast evolving games, with new cards and game rules coming out multiple times a year. Since the late 1990s, CCGs have been turned into computer games. New versions of those computer games however, are released on a less regular basis. One of the reasons for this is that for every new card or game rule, new source code has to be written.

In an ideal world where computers are able to completely understand natural language, new cards could be added to the computer game on the fly by the author of the cards, only using the instructions printed on the cards. A solution right in the middle between writting actual source code an parsing and interpreting natural language is a Domain Specific Language (DSL). A DSL is a programming language designed to solve problems within one well-defined set of problems. This article explains how a DSL is be used to counter the fast evolving game mechanics of collectible card games.

Section \ref{CCGs} describes CCGs, their digital versions and why they are in need of an easy way to implement frequent changes. Section \ref{DSLs} explains the concept of DSLs and Section \ref{Scala} is a short description of the programming language used to embed the DSL in: Scala.
Our solution is explained in detail in Section \ref{Solution} and evaluated in Section \ref{Evaluation} against different types of possible changes in game mechanics.
